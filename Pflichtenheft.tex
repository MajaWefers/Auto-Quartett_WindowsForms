\documentclass[11pt]{scrreprt}
\usepackage[utf8]{inputenc}
\usepackage{graphicx}
\begin{document}

\begin{center}
\Huge
Auto-Quartett

\medskip
\large
Realisierung eines Spiels 

\vskip 1in
\Huge
Pflichtenheft

\large
\vskip 2in
Auftraggeber:\\
Herr Mango, Herr Lackmann\\
Entwicklungszeitraum:\\
17.04.2017 bis 12.06.2017

\large
\vskip 2in
Vorgelegt von:\\
Maja Blömer (MB)\\
Rajeethan Dhayaparan (RD)\\
Vitali Müller (VM)

\end{center}

\newpage
\tableofcontents

\newpage
\chapter{Einleitung}

\section{Notwendigkeit der Software}
\normalsize \normalfont \textnormal
...

\section{Zielbeschreibung}
Es soll ein Programm erstellt werden, das ein Auto-Quartett-Spiel simuliert. Dieses soll Schülern als Lernmaterial zum Nachvollziehen von Arrays und Strukturen dienen. Zu diesem Zweck ist der Code entsprechend zu dokumentieren..


\newpage

\section{Entwurfsziele}
\paragraph{Korrektheit:} Die Aufgaben aller definierten Funktionen sollen wie vorgesehen erfüllt werden.

\paragraph{Zuverlässigkeit:} Ein problemloses Starten und Laufen der Software soll stets gewährleistet sein.

\paragraph{Verständlichkeit:} Durch eine ausführliche und eindeutige Dokumentation, sowie kommentierter Quellcode soll die Software leichter zu verstehen sein.

\paragraph{Performanz:} Die Software soll die üblichen Browsing-Gewohnheiten des Nutzers nicht beeinflussen.

\paragraph{Benutzerfreundlichkeit:} Die Bedienung der Software soll größtenteils selbsterklärend sein. Ein Benutzerhandbuch soll überflüssig sein.

\paragraph{Konsistenz} Die Software soll zu jedem Zeitpunkt korrekte Ergebnisse liefern.(BESPRECHEN MIT TEAM!!!)Im Falle von Problemen und Fehlern soll eine Fehlermeldung, statt falscher Ergebnisse geliefert werden.

\section{Muss-Kriterien}
\begin{itemize}
\item \textbf{M1}: Es soll eine Konsolenanwendung erstellt werden, die ein Auto-Quartett-Spiel simuliert.
    \begin{itemize}
    \item ...
    \item ...
    \item ... 
    \end{itemize}
\item \textbf{M2}: Drei Karten sind bereits durch das Lastenhaft vorgegeben
    \begin{itemize}
    \item ...
    \item ...
    \item ... 
    \end{itemize}
\item \textbf{M3}: Eine zusätzliche vierte Karte soll mit Hilfe einer Eingabemaske erfasst werden können.
    \begin{itemize}
     \item ...
    \item ...
    \item ... 
    \end{itemize}
\item \textbf{M4}: Der Quellcode soll auf der Basis von allgemeinen Kontrollstrukturen, Arrays und Strukturen aufgebaut werden.\textbackslash Bedienung
    \begin{itemize}
    \item ...
    \item ...
    \item ... 
    \end{itemize}
\item \textbf{M5}: Je nach Benutzereingabe soll die Ausgabemaske Folgendes anzeigen
    \begin{itemize}
    \item ...
    \item ...
    \item ... 
    \end{itemize}
\item \textbf{M6}: Die Ein- und Ausgabe und Logik-Funktionen sollen voneinander getrennt sein.
    \begin{itemize}
    \item ...
    \item ...
    \item ... 
    \end{itemize}
\item \textbf{M7}: Der Code soll so kommentiert werden, dass er von Schülern nachvollzogen werden kann.
    \begin{itemize}
    \item ...
    \item ...
    \item ... 
    \end{itemize}
    
    
\end{itemize}



\section{Wunsch-Kriterien}
\begin{itemize}
\item \textbf{W1}: Es soll eine Konsolenanwendung erstellt werden, die ein Auto-Quartett-Spiel simuliert.
    \begin{itemize}
    \item ...
    \item ...
    \item ... 
    \end{itemize}
\item \textbf{W2}: Drei Karten sind bereits durch das Lastenhaft vorgegeben
    \begin{itemize}
    \item ...
    \item ...
    \item ... 
    \end{itemize}
\item \textbf{W3}: Eine zusätzliche vierte Karte soll mit Hilfe einer Eingabemaske erfasst werden können.
    \begin{itemize}
     \item ...
    \item ...
    \item ... 
    \end{itemize}
\item \textbf{W4}: Der Quellcode soll auf der Basis von allgemeinen Kontrollstrukturen, Arrays und Strukturen aufgebaut werden.\textbackslash Bedienung
    \begin{itemize}
    \item ...
    \item ...
    \item ... 
    \end{itemize}
\item \textbf{W5}: Je nach Benutzereingabe soll die Ausgabemaske Folgendes anzeigen
    \begin{itemize}
    \item ...
    \item ...
    \item ... 
    \end{itemize}
\item \textbf{W6}:Die Ein- und Ausgabe und Logik-Funktionen sollen voneinander getrennt sein.
    \begin{itemize}
    \item ...
    \item ...
    \item ... 
    \end{itemize}
\item \textbf{W7}: Der Code soll so kommentiert werden, dass er von Schülern nachvollzogen werden kann.
    \begin{itemize}
    \item ...
    \item ...
    \item ... 
    \end{itemize}
\item \textbf{W8}: Der Code soll so kommentiert werden, dass er von Schülern nachvollzogen werden kann.
    \begin{itemize}
    \item ...
    \item ...
    \item ... 
    \end{itemize}
\item \textbf{W9}: Der Code soll so kommentiert werden, dass er von Schülern nachvollzogen werden kann.
    \begin{itemize}
    \item ...
    \item ...
    \item ... 
    \end{itemize}
\item \textbf{W10}: Der Code soll so kommentiert werden, dass er von Schülern nachvollzogen werden kann.
    \begin{itemize}
    \item ...
    \item ...
    \item ... 
    \end{itemize}
\item \textbf{W11}: Der Code soll so kommentiert werden, dass er von Schülern nachvollzogen werden kann.
    \begin{itemize}
    \item ...
    \item ...
    \item ... 
    \end{itemize}
    
\end{itemize}

\section{Abgrenzungskriterien}
\begin{itemize}
\item Die Software bietet ausschließlich das Spielen von Auto-Quartett an.
\item Der Benutzer muss selbst neue Karten hinzufügen die automatisch in einer .xml-Datei gespeichert werden.
\item Der Benutzer kann ausschließlich nur Auto Karten erstellen  mit den vorgegebenen Attributen. 
\item Die Software bietet nur, dass spielen gegen den Computer oder eine Person an.  
\item Die Software ist nur mit der selbst erzeugten .xml-Datei kompatibel.
\end{itemize}

\section{Funktionsbeschreibungen}
\paragraph{...}


\paragraph{...}


\chapter{Vorgehensmodell – Wasserfallmodell}
\section{Allgemeine Definition Wasserfallmodell}

...


\newpage

\chapter{Verwendete Technologien und Entwicklungswerkzeuge}
\section{Codeverwaltung}
\paragraph{Git \& GitHub}
Der Sourcecode des Projektes wird über Git verwaltet, welches wiederum über das Portal GitHub.com für jeden als freie Software zur Verf\"ugung gestellt wird.
Git ermöglicht es allen, gleichzeitig an der Software zu arbeiten und die \"Anderungen im Anschluss wieder zusammen zu f\"uhren um eine ganzheitliche Software zu erstellen. Sie bietet außerdem eine vollst\"andige History, womit Fehler zu jeder Zeit r\"uckg\"angig gemacht werden k\"onnen.
Des weiteren ist vollst\"andig nachvollziehbar, welche Person zu welchem Zeitpunkt an welchem St\"uck Quellcode gearbeitet hat.

\section{Kommunikation und Organisation}
\paragraph{Gitter.com}
Zur Kommunikation wird haupts\"achlich Gitter genutzt, welches sehr stark mit GitHub.com verkn\"upft ist und somit eine sehr stark projektorientierte Kommunikationsplattform. Sie bietet neben spezieller Textformatierung f\"ur Quellcode auch die Referenzierung auf Personen, Tickets und auch Dateien im Github Projekt. Außerdem sind auf der Plattform aktuelle Aktivit\"aten des Projektes \"uber sogenannte "Git-Hooks" ersichtlich.
\paragraph{WhatsApp}
Zur Organisation von Terminen, Versp\"atungen u.ä. dient außerdem die bekannte Messaging-Applikation WhatsApp.


\section{Programmiersprachen}
\paragraph{C\#}
...
\paragraph{XML}
...
\newpage
\chapter{Umsetzung}
\section{Menü (VM)}
...
\section{Eingabe (RD)}
...
\section{Ausgabe (MB)}
...
\section{GUI und Steuerung durch Benutzer)}
...
\newpage
\chapter{Anhänge}
\section{Struktogramm für Menüstruktur und Programmablauf (MB)}
Im folgenden Struktogramm wird ein grobes Schema dargestellt, wie das Spiel über die Menüstruktur in etwa gesteuert wird.

\newpage
\chapter{Glossar}
\paragraph{TETS} ...
\end{document}
